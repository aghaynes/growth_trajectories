\documentclass[10pt,twoside]{article}\usepackage[]{graphicx}\usepackage[]{color}
%% maxwidth is the original width if it is less than linewidth
%% otherwise use linewidth (to make sure the graphics do not exceed the margin)
\makeatletter
\def\maxwidth{ %
  \ifdim\Gin@nat@width>\linewidth
    \linewidth
  \else
    \Gin@nat@width
  \fi
}
\makeatother

\definecolor{fgcolor}{rgb}{0.345, 0.345, 0.345}
\newcommand{\hlnum}[1]{\textcolor[rgb]{0.686,0.059,0.569}{#1}}%
\newcommand{\hlstr}[1]{\textcolor[rgb]{0.192,0.494,0.8}{#1}}%
\newcommand{\hlcom}[1]{\textcolor[rgb]{0.678,0.584,0.686}{\textit{#1}}}%
\newcommand{\hlopt}[1]{\textcolor[rgb]{0,0,0}{#1}}%
\newcommand{\hlstd}[1]{\textcolor[rgb]{0.345,0.345,0.345}{#1}}%
\newcommand{\hlkwa}[1]{\textcolor[rgb]{0.161,0.373,0.58}{\textbf{#1}}}%
\newcommand{\hlkwb}[1]{\textcolor[rgb]{0.69,0.353,0.396}{#1}}%
\newcommand{\hlkwc}[1]{\textcolor[rgb]{0.333,0.667,0.333}{#1}}%
\newcommand{\hlkwd}[1]{\textcolor[rgb]{0.737,0.353,0.396}{\textbf{#1}}}%

\usepackage{framed}
\makeatletter
\newenvironment{kframe}{%
 \def\at@end@of@kframe{}%
 \ifinner\ifhmode%
  \def\at@end@of@kframe{\end{minipage}}%
  \begin{minipage}{\columnwidth}%
 \fi\fi%
 \def\FrameCommand##1{\hskip\@totalleftmargin \hskip-\fboxsep
 \colorbox{shadecolor}{##1}\hskip-\fboxsep
     % There is no \\@totalrightmargin, so:
     \hskip-\linewidth \hskip-\@totalleftmargin \hskip\columnwidth}%
 \MakeFramed {\advance\hsize-\width
   \@totalleftmargin\z@ \linewidth\hsize
   \@setminipage}}%
 {\par\unskip\endMakeFramed%
 \at@end@of@kframe}
\makeatother

\definecolor{shadecolor}{rgb}{.97, .97, .97}
\definecolor{messagecolor}{rgb}{0, 0, 0}
\definecolor{warningcolor}{rgb}{1, 0, 1}
\definecolor{errorcolor}{rgb}{1, 0, 0}
\newenvironment{knitrout}{}{} % an empty environment to be redefined in TeX

\usepackage{alltt}

\usepackage{suppmat}
\usepackage{listings}
\usepackage[T1]{fontenc} % Better underscores, shockingly.



\usepackage[rgb,dvipsnames]{xcolor}
\definecolor{grey}{rgb}{0.5, 0.5, 0.5}

\newcommand{\smurl}[1]{\url{#1}}
\newcommand{\ud}{\ensuremath{\rm{d}}}
\newcommand{\tabitem}{~~\llap{\textbullet}~~}
\newcommand{\email}[1]{\href{mailto:#1}{\texttt{#1}}}


\title{Untangling the link between traits, size and growth rate in plants}
\runninghead{I\lowercase{nfluence of traits on growth}}

\author{Daniel S. Falster\textasteriskcentered \& Richard G. FitzJohn}

\affiliation{Biological Sciences, Macquarie University NSW 2109, Australia
\\
\textasteriskcentered Correspondence author. E-mail: \texttt{adaptive.plant@gmail.com}
}

\usepackage{natbib}
\bibliographystyle{ecol_let}
\setcitestyle{authoryear,open={(},close={)}}
% allows for numbered referencing, as required by ecol letters
\usepackage{etoolbox}
\newbool{MyRefNumbers}
\booltrue{MyRefNumbers} % comment to remove numbers in reference list


\usepackage{graphicx}
\graphicspath{{../output/}}

% We will generate all images so they have a width \maxwidth. This means
% that they will get their normal width if they fit onto the page, but
% are scaled down if they would overflow the margins.
\makeatletter
\def\maxwidth{\ifdim\Gin@nat@width>\linewidth\linewidth
\else\Gin@nat@width\fi}
\makeatother

\let\Oldincludegraphics\includegraphics
\renewcommand{\includegraphics}[1]{\Oldincludegraphics[width=\maxwidth]{#1}}


% We will generate all images so they have a width \maxwidth. This means
% that they will get their normal width if they fit onto the page, but
% are scaled down if they would overflow the margins.
\makeatletter
\def\maxwidth{\ifdim\Gin@nat@width>\linewidth\linewidth\else\Gin@nat@width\fi}
\def\maxheight{\ifdim\Gin@nat@height>\textheight\textheight\else\Gin@nat@height\fi}
\makeatother
\setkeys{Gin}{width=\maxwidth,height=\maxheight,keepaspectratio}

\titleprefix{Supporting Materials for:}

\date{}
\IfFileExists{upquote.sty}{\usepackage{upquote}}{}
\begin{document}

\maketitle

\tableofcontents

\renewcommand{\thefigure}{S\arabic{figure}}
\renewcommand{\thetable}{S\arabic{table}}
\setcounter{secnumdepth}{0}

\clearpage

\section{Supporting Tables}\label{app:supp_info_table}



\begin{table}[ht]
\caption{Model parameters}
\centering
{\footnotesize  % smaller font, double space
\begin{doublespace}
% \include{table-pars}

\end{doublespace}
}
\label{tab:params}
\end{table}

\newpage

\section{Supporting Figures}\label{app:supp_info_figures}


\begin{figure}[ht]
\centering
\includegraphics{SI/lma_tradeoff.pdf}
\caption{\textbf{Leaf turnover decreases with leaf-mass per unit leaf area.}
Data from \citep{Wright-2004} for 678 species from 51 sites, each
point giving a species-average. Lines show standardised major axis lines
fitted to datsa from each site, with intensity of shading adjusted
according to strength of the relationship.\label{fS-leaf}}
\end{figure}

\newpage


\begin{figure}[ht]
\centering
\includegraphics{SI/growth_light_dia.pdf}
\caption{\textbf{Traits moderate the responsiveness of growth to changes
in light environment (stem-diameter growth).} Panels show predicted relationship between specific trait and diameter growth rate, for a plant of specified
diameter and under a range of shading environments.
\label{fig:growth_light_dia}}
\end{figure}

\newpage


\begin{figure}[ht]
\centering
\includegraphics{SI/growth_light_area.pdf}
\caption{\textbf{Traits moderate the responsiveness of growth to changes
in light environment (stem-area growth).} Panels show predicted relationship between specific trait and diameter growth rate, for a plant of specified
diameter and under a range of shading environments.
\label{fig:growth_light_area}}
\end{figure}


\newpage

\begin{figure}[ht]
\centering
\includegraphics{SI/growth_light_mass.pdf}
\caption{\textbf{Traits moderate the responsiveness of growth to changes
in light environment (mass growth).} Panels show predicted relationship between
specific trait and mass growth rate, for a plant of specified
mass and under a range of shading environments.
\label{fig:growth_light_mass}}
\end{figure}

\newpage



\clearpage

\bibliography{references}

\end{document}

